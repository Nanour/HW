\documentclass{article}
\usepackage[utf8]{inputenc}
\usepackage[margin=1.2in]{geometry}
\usepackage{graphicx}
\usepackage{enumitem}
\usepackage{adjustbox}
\usepackage{hyperref}
\usepackage{framed}

\def\answer#1{
    \vspace*{0.5em}\\
    \noindent\fbox{%
    \parbox{\textwidth}{%
		#1
    }%
}
}

\newcommand{\mycoursenum}{15-440/15-640}
\newcommand{\myhwnum}{3}
\newcommand{\myname}{}   % put your name here
\newcommand{\myandrew}{}  % put your andrewID here

\newif\ifprint
\printfalse  % change to printfalse after adding your name and andrewID

\def\changemargin#1#2{\list{}{\rightmargin#2\leftmargin#1}\item[]}
\let\endchangemargin=\endlist 

\renewcommand{\labelenumii}{\theenumii}
\renewcommand{\theenumii}{\theenumi.\arabic{enumii}.}

\begin{document}
\medskip
\thispagestyle{plain}
\begin{center}
{\Large \mycoursenum: Homework \myhwnum}\\
Due: November 10, 2016 10:30am \\
\medskip
\begin{tabular}{| l p{3in}|}
\hline
Name: \myname& \\
& \\ \hline
Andrew ID: \myandrew & \\
&\\
\hline
\end{tabular}
\end{center}



%%%%%%%%%%%%%%%%%%%%%%%%%%%%%%%%%%%%%%%%%%%%%%%%%%%%%%%%%%%%%%%%%%%%%%%%%%%%%%%%%%

\section{Hadoop (12 points)}
\begin{enumerate}
\item Harry Bovik has just got access to a fancy GHC cluster (100 nodes) and decides to utilize every last resource available by launching a massive Hadoop job that spawns thousands of Map and Reduce tasks. However, to his disappointment, he finds that this job takes longer time to complete than he had expected from this cluster. He is later informed by the cluster admin that a couple of nodes use Intel Xeon processors that were released all the way back in 2007 and need to be upgraded to the latest Skylake Xeons that the rest of the nodes use. Why does he observe this issue, even though just 2 out of the 100 nodes in the cluster are sluggish? \textbf{[4 points]}
% \begin{framed}
% 	Insert answer here
% \end{framed}

\item Harry has no time to wait for this hardware upgrade to happen and uses the knowledge he gained from 15-440 to arrive at a couple of solutions that he can directly implement on Hadoop. What are they? \textbf{[4 points]}
% \begin{framed}
% 	Insert answer here
% \end{framed}

\item After using the above solutions, Harry is still disappointed with the performance of his job. He profiles the cluster and discovers that the network I/O between HDFS nodes is the biggest bottleneck. He understands the problem and decides to write his own Hadoop scheduler. What was the problem and why does Harry make this decision? \textbf{[4 points]}
% \begin{framed}
% 	Insert answer here
% \end{framed}

\end{enumerate}

\section{Iterative MapReduce (20 Points)}
Sally has a very very very large map. She’s really excited about finding all the movie theatres on the map, and how long it would take to get to each one (she collects movie tickets from various theatres,
it’s quite an impressive collection). Sally will need your help to find the shortest path from her house to every movie theatre using MapReduce.
Assume the following:
- There are arbitrary points on the graph that represent intersections, known as A, B, C, etc.
- All edges on the graph are undirected.

Clearly, on a graph of smaller size, Sally can use Dijkstra’s to find the information she needs, but because this graph is so large, she wants to use MapReduce to run a parallelized single-source shortest path algorithm. She designs a algorithm that consists of two phases
\begin{itemize}
\item The first phase reads and processes the input adjacency list of the graph (in the same format as shown above)
\item The second phase receives the processed output from the first phase and runs multiple iterations of the \textit{same} Map and Reduce code to finally output the shortest path to each theatre from Sally's location once the algorithm converges.
\end{itemize}

Hints - 
\begin{itemize}
\item Trace the possible steps using this sample adjacency list input. Note that the output of a Reduce phase is the input for the subsequent Map phase.
\item The values in \verb|<Key, Value>| pairs for the Map input/output aren't limited to discrete data types (like int, floats, strings, etc). Permitted complex data types are lists and tuples and nestings of these two data types.
\item The input for Reduce is always a key followed by the list of all values emitted by Map for that key.
\end{itemize}

\begin{enumerate}
\item What would be the steps in the first Map phase? (Pseudo-code or high-level design logic will suffice) \textbf{[2 Points]}
% \begin{framed}
% 	Insert answer here
% \end{framed}

\item What are the inputs and outputs for the first Map phase? (express as \verb|<Key, Value>|.)\\ \textbf{[3 Points]}
% \begin{framed}
% 	Insert answer here
% \end{framed}

\item What would be the steps in the first Reduce phase? (Pseudo-code or high-level design logic will suffice) \textbf{[2 Points]}
% \begin{framed}
% 	Insert answer here
% \end{framed}

\item What are the inputs and outputs for the first Reduce phase? (express as \verb|<Key,List of Values>|.) \textbf{[3 Points]}
% \begin{framed}
% 	Insert answer here
% \end{framed}

\item What would be the steps in the second Map phase? (Pseudo-code or high-level design logic will suffice) \textbf{[2 Points]}
% \begin{framed}
% 	Insert answer here
% \end{framed}

\item What are the inputs and outputs for the second Map phase? (express as \verb|<Key,Value>|.)\\ \textbf{[3 Points]}
% \begin{framed}
% 	Insert answer here
% \end{framed}

\item What would be the steps in the second Reduce phase? (Pseudo-code or high-level design logic will suffice) \textbf{[2 Points]}
% \begin{framed}
% 	Insert answer here
% \end{framed}

\item What are the inputs and outputs for the second Reduce phase? (express as \verb|<Key,List of Values>|.) \textbf{[3 Points]}
% \begin{framed}
% 	Insert answer here
% \end{framed}

\end{enumerate}
%%%%%%%%%%%%%%%%%%%%%%%%%%%%%%%%%%%%%%%%%%%%%%%%%%%%%%%%%%%%%%%%%%%%%%%%%%%%%%%%%%
\section{DNS (14 Points)}

\begin{enumerate}
\item Elisa wants to listen to the National Public Radio news over the Internet. She starts her favorite
audio player and points it to ra1.streaming.npr.org. The audio player calls gethostbyname()
with the given name to obtain the IP address of the server. As a result of the gethostbyname()
call, the local resolver in Elisa's machine contacts the local DNS server to translate the host name
into an IP address. The local DNS server performs an iterative lookup.
The table below contains the DNS distributed database. A row corresponds to a DNS record. The
records are grouped by DNS server.

\vspace{0.1in}
\begin{center}
\begin{tabular}{|l|l|l|l|l|l|}
\hline
{\normalsize Record} & Name & TTL & IN & Type & Value\\
\# &      & (sec) & & & \\
\hline
\multicolumn{6}{|c|}{\texttt{localdns.localdomain.com}} \\
\hline
R1 & . & 262542 & IN & NS & \texttt{E.ROOT-SERVERS.NET.} \\
R2 & \texttt{E.ROOT-SERVERS.NET.} & 348942 & IN& A & 192.203.230.10 \\
\hline
\multicolumn{6}{|c|}{\texttt{E.ROOT-SERVERS.NET}} \\
\hline
R3 & org. & 172800 & IN& NS & \texttt{F.GTLD-SERVERS.NET} \\
R4 & \texttt{F.GTLD-SERVERS.NET} & 172800 & IN& A & 192.35.51.30 \\
\hline
\multicolumn{6}{|c|}{\texttt{F.GTLD-SERVERS.NET}} \\
\hline
R5 & \texttt{npr.org} & 172800 & IN & NS & \texttt{watson.npr.org.} \\
R6 & \texttt{watson.npr.org.} & 172800 & IN & A & 205.153.37.175 \\
\hline
\multicolumn{6}{|c|}{\texttt{watson.npr.org}} \\
\hline
R7 & \texttt{streaming.npr.org.} & 172800 & IN & NS &
\texttt{ns.streaming.npr.org.} \\
R8 & \texttt{ns.streaming.npr.org} & 172800 & IN & A & 205.153.36.175
\\
\hline
\multicolumn{6}{|c|}{\texttt{ns.streaming.npr.org}} \\
\hline
R9 & \texttt{audio.streaming.npr.org.} & 172800 & IN & CNAME &
\texttt{ra1.streaming.npr.org.} \\
R10 & \texttt{ra1.streaming.npr.org.} & 10 & IN & A & 205.153.36.175 \\
\hline
\end{tabular}
\end{center}

Attach the figure from the handout and draw arrows to indicate the sequence of queries and responses exchanged among the different machines. Label each arrow with a sequence number, and fill in the table below to indicate the following information: 

\begin{itemize}
\item Sequence number indicating the ordering of the message exchanges.
\item Message Type: use Q for Query or R for Response.
\item Data: For queries use the value of the question data. 
For responses, specify the record ID(s) returned, if any, from the 
first column in Figure 1, e.g., R1, R2 ....
\item You may use abbreviations for host names, e.g. ``ra1'' rather than ra1.streaming.npr.org.
\end{itemize}

The figure in the handout already contains an arrow indicating the first message from the local resolver to the local DNS server. The sequence number is 1 (first message), type = Q (query) and the data is the host name the application wants to resolve (ra1.streaming.npr.org). To make your sequence as simple as possible, assume the server includes both the A and NS records when applicable, so include both of them in the corresponding message. \textbf{[8 Points]}

\vspace{0.1in}
\begin{table}[h]
\centering
\begin{adjustbox}{width=8cm}
\begin{tabular}{lll}
Seq & Type & Data \\
\hline
1 & Q & ra1.streaming.npr.org(A)
\end{tabular}
\end{adjustbox}
\end{table}

%\vspace{0.1in}
%\includegraphics[width=.8\textwidth]{Insert Solution image file here}
%\vspace{0.4in}
% \begin{framed}
% 	Insert answer here
% \end{framed}

\item Eliza repeats her query two minutes later.  Show what happens for this subsequent query.\\ \textbf{[6 points]}
% \begin{framed}
% 	Insert answer here
% \end{framed}

\end{enumerate}
%%%%%%%%%%%%%%%%%%%%%%%%%%%%%%%%%%%%%%%%%%%%%%%%%%%%%%%%%%%%%%%%%%%%%%%%%%%%%%%%%%
\section{DIG (18 points)}
In this problem, we will use dig tool available on Linux and Mac OS to explore DNS servers. You can
read about it using \verb|man dig|. Recall that a DNS server higher in the DNS hierarchy delegates a DNS
query to a DNS server lower in the hierarchy, by sending back to the DNS client the name of that
lower-level DNS server (assuming no recursion is specified). \textbf{For each of the following questions, show the sequence of commands that you ran on your shell with dig and the output they generate.} \textit{Hint:} Be sure to use the \verb|+norecurse| option
to dig, and remember that you will need to specify different target DNS servers (@) each time. Your
queries should look like:\\
\verb|dig +norecurse @< targetserver> RecordToResolve RecordType|.

\begin{enumerate}
\item Starting with a root DNS server (from one of the root servers [a-m].root-servers.net),  initiate a
sequence of queries using dig for the A-type record for www.blog.xkcd.com without using recursion.
Be sure to also show the list of the names of DNS servers in the entire delegation chain starting from
the root in answering your query. \textbf{[5 points]}
% \begin{framed}
% 	Insert answer here
% \end{framed}

\item  Repeat the same procedure as above for www.csd.cs.cmu.edu \textbf{[5 points]}
% \begin{framed}
% 	Insert answer here
% \end{framed}

\item Repeat the same procedure for 81.183.132.209.in-addr.arpa. This time, dig for the PTR-type
record. \textbf{[5 points]}
% \begin{framed}
% 	Insert answer here
% \end{framed}

\item Use dig -x IP addr to perform reverse DNS lookup for the IP address 209.132.183.81. What
is the domain name associated with this IP address? What is the type of the DNS record in the
answer section? Compare the answer section to part 3’s answer. \textbf{[3 points]}
% \begin{framed}
% 	Insert answer here
% \end{framed}

\end{enumerate}

%%%%%%%%%%%%%%%%%%%%%%%%%%%%%%%%%%%%%%%%%%%%%%%%%%%%%%%%%%%%%%%%%%%%%%%%%%%%%%%%%%

\section{Web and Peer-to-Peer (14 points)}
\begin{enumerate}
\item{Provide three reasons a company might prefer to pay Akamai to host their
webpage instead of putting it onto a peer-to-peer network (such as Napster) for
free. \textbf{[6 points]}}
% \begin{framed}
% 	Insert answer here
% \end{framed}

\item{We saw no examples of of chunk-based peer-to-peer networks that use
flooding. What would make such a network inefficient? \textbf{[4 points]}}
% \begin{framed}
% 	Insert answer here
% \end{framed}

\item{While not strictly true, people tend to view hash tables as offering
constant time look up in practice. (The possibility of a bad hash function
leading to many collisions is why it is not strictly constant time.)
Distributed Hash Tables (DHTs), on the other hand, only offer $O(\log n)$-time
lookup where $n$ is the number of nodes in the DHT.  This is odd, since the two
appear to be equivalent, i.e. simply map each entry in a traditional hash table
onto a node in the DHT.  What property or requirement of a DHT makes this
approach impractical in practice. \textbf{[4 points]}}
% \begin{framed}
% 	Insert answer here
% \end{framed}

\end{enumerate}

%%%%%%%%%%%%%%%%%%%%%%%%%%%%%%%%%%%%%%%%%%%%%%%%%%%%%%%%%%%%%%%%%%%%%%%%%%%%%%%%%%

\section{Consistent Hashing (12 points)}
David is designing a distributed hash table with n nodes. The table will store
values with m-bit keys; each node in the DHT has an ID obtained by hashing the
nodes name (e.g., ID1 = h(”node1”)). He is considering two schemes for
assigning key-value pairs to nodes responsible for storing them:

\begin{itemize}
\item{\textbf{Scheme 1}: Use consistent hashing. The node responsible for key k is the first node whose ID is equal to or follows k in the identifier space (modulo 2m).}
\item{\textbf{Scheme 2}: Order the nodes numerically by their IDs. If a nodes
position in this ranking is $r$ (where $r \in [0, n)$), it is responsible for
keys in the range $[r\cdot 2m, (r+1)\cdot 2m)$.}
\end{itemize}

\begin{enumerate}
\item{What is one advantage of scheme 1? \textbf{[3 points]}}
% \begin{framed}
% 	Insert answer here
% \end{framed}

\item{What is one advantage of scheme 2? \textbf{[3 points]}}
% \begin{framed}
% 	Insert answer here
% \end{framed}

\item{Consider a new node joining the DHT (for a new total of n+ 1). On
average, for each scheme, what fraction of the key space will be assigned to a
different node as a result? For simplicity, assume the following:
\begin{itemize}
\item{The new nodes ID is higher than any current nodes ID.}
\item{The IDs of the current nodes are evenly distributed: $ID_i = \frac{i}{n}
* 2^m$} \textbf{[6 points]}
\end{itemize}}
% \begin{framed}
% 	Insert answer here
% \end{framed}

\end{enumerate}

%%%%%%%%%%%%%%%%%%%%%%%%%%%%%%%%%%%%%%%%%%%%%%%%%%%%%%%%%%%%%%%%%%%%%%%%%%%%%%%%%%

\section{Chord and DHT (10 points)}
Skype uses a custom protocol to determine whether a user is logged
in, where they are located, and what ports they are listening on.
Suppose that you have set out to build your own peer-to-peer telephony
system.

\begin{enumerate}
\item{Briefly, how could you use the Chord distributed hash table to
maintain a directory of users that was stored entirely on the end-users
computers, with no centralized infrastructure? \textbf{[4 points]}}
% \begin{framed}
% 	Insert answer here
% \end{framed}

\item{Your telephony system catches on very quickly. Soon you have 100,000
users worldwide. However, users start to complain that it takes a long time to
perform directory lookups. Assume that the average latency between users is
100ms. Explain why are lookups taking so long and estimate how long they take.
\textbf{[6 points]}}
% \begin{framed}
% 	Insert answer here
% \end{framed}

\end{enumerate}	
\end{document}

